\documentclass[useAMS,usenatbib]{mn2e}
\input psfig.sty
\usepackage{array}
%\usepackage{graphicx}
%\usepackage{epstopdf}

\voffset=-0.8in

\def \gas {\textsc{gasoline}}
\def \gastwo {\textsc{ESF-gasoline2}}
\def \changa {\textsc{changa}}
\def \mean#1{\left< #1 \right>}

%%MISC
\def \aj {AJ}
\def \apj {ApJ}
\def \apjl {ApJL}
\def \mnras {MNRAS}
\def \apjs {ApJS}
\def \aap {A\&A}
\def \nat {Nature}
\def \pasp {PASP}
\def \na {NewA}

\def \etal {et~al.~}
\def \eg{e.g.}
\def \Section{\S}
\def \spose#1{\hbox  to 0pt{#1\hss}}  
\def \lta{\mathrel{\spose{\lower 3pt\hbox{$\sim$}}\raise  2.0pt\hbox{$<$}}}
\def \gta{\mathrel{\spose{\lower  3pt\hbox{$\sim$}}\raise 2.0pt\hbox{$>$}}}
\def \ion#1#2{#1{\footnotesize{#2}}\relax}
\def \ha       {H$\alpha$}
\def \hi       {\ion{H}{I}}
\def \hii      {\ion{H}{II}}
\def \oii      {[\ion{O}{II}]}

%%UNITS
\def \kmsmpc {\>{\rm km}\,{\rm s}^{-1}\,{\rm Mpc}^{-1}}
\def \kms {\ifmmode  \,\rm km\,s^{-1} \else $\,\rm km\,s^{-1}  $ \fi }
\def \kpc {\ifmmode  {\rm kpc}  \else ${\rm  kpc}$ \fi  }  
\def \hkpc {\ifmmode  {h^{-1}\rm kpc}  \else ${h^{-1}\rm kpc}$ \fi  }  
\def \hMpc {\ifmmode  {h^{-1}\rm Mpc}  \else ${h^{-1}\rm Mpc}$ \fi  }  
\def \Msun {\ifmmode \rm M_{\odot} \else $\rm M_{\odot}$ \fi}
\def \hMsun {\ifmmode h^{-1}\,\rm M_{\odot} \else $h^{-1}\,\rm M_{\odot}$ \fi}
\def \hhMsun {\ifmmode h^{-2}\,\rm M_{\odot}\else $h^{-2}\,\rm M_{\odot}$ \fi}
\def \Lsun {\ifmmode L_{\odot} \else $L_{\odot}$ \fi} 
\def \hhLsun {\ifmmode h^{-2}\,\rm L_{\odot} \else $h^{-2}\,\rm L_{\odot}$ \fi}

%%COSMOLOGY

\def\LCDM{$\Lambda$CDM }
\def \LCDM {\ifmmode \Lambda{\rm CDM} \else $\Lambda{\rm CDM}$ \fi}
\def \sig8 {\ifmmode \sigma_8 \else $\sigma_8$ \fi} 
\def \Omegam {\ifmmode \Omega_{\rm m} \else $\Omega_{\rm m}$ \fi} 
\def \Omegab {\ifmmode \Omega_{\rm b} \else $\Omega_{\rm b}$ \fi} 
\def \Omegar {\ifmmode \Omega_{\rm r} \else $\Omega_{\rm r}$ \fi} 
\def \fbar {\ifmmode f_{\rm bar} \else $f_{\rm bar}$ \fi} 
\def \OmegaL {\ifmmode \Omega_{\rm \Lambda} \else $\Omega_{\rm \Lambda}$\fi} 
\def \Deltavir {\ifmmode \Delta_{\rm vir} \else $\Delta_{\rm vir}$ \fi}
\def \rhocrit {\ifmmode \rho_{\rm crit} \else $\rho_{\rm crit}$ \fi}

%DARK MATTER
\def \rs {\ifmmode r_{\rm s} \else $r_{\rm s}$ \fi} 
\def \rh {\ifmmode r_{\rm h} \else $r_{\rm h}$ \fi} 
\def \Rvir {\ifmmode R_{\rm vir} \else $R_{\rm vir}$ \fi}
\def \Vvir {\ifmmode V_{\rm  vir} \else  $V_{\rm vir}$  \fi} 
\def \Vmax {\ifmmode V_{\rm  max} \else  $V_{\rm max}$  \fi} 
\def \Mvir {\ifmmode M_{\rm  vir} \else $M_{\rm  vir}$ \fi}  
\def \Mhalo {\ifmmode M_{\rm halo} \else $M_{\rm  halo}$ \fi}  
\def \Nvir {\ifmmode N_{\rm  vir} \else $N_{\rm  vir}$ \fi}  
\def \Jvir {\ifmmode J_{\rm vir} \else $J_{\rm vir}$ \fi} 
\def \Evir {\ifmmode E_{\rm vir} \else $E_{\rm vir}$ \fi} 
\def \lam {\ifmmode \lambda  \else $\lambda$ \fi} 
\def \lamp {\ifmmode \lambda^{\prime} \else $\lambda^{\prime}$  \fi} 
\def \lampc {\ifmmode \lambda^{\prime}_{\rm c} \else
  $\lambda^{\prime}_{\rm c}$  \fi} 

\def \xoff {\ifmmode x_{\rm off} \else $x_{\rm off}$ \fi}
\def \rhorms {\ifmmode \rho_{\rm rms} \else $\rho_{\rm rms}$ \fi}
\def \qbar {\ifmmode \bar{q} \else $\bar{q}$ \fi}

%%BARYONS
\def \Mb {\ifmmode M_{\rm b} \else $M_{\rm b}$ \fi} 
\def \eSF {\ifmmode \epsilon_{\rm SF} \else $\epsilon_{\rm SF}$ \fi} 
\def \Md {\ifmmode M_{\rm d} \else $M_{\rm d}$ \fi} 
\def \Mg {\ifmmode M_{\rm g} \else $M_{\rm g}$ \fi} 
\def \Rb {\ifmmode R_{\rm b} \else $R_{\rm b}$ \fi} 
\def \Rd {\ifmmode R_{\rm d} \else $R_{\rm d}$ \fi} 
\def \Rg {\ifmmode R_{\rm g} \else $R_{\rm g}$ \fi} 
\def \mgal {\ifmmode m_{\rm gal} \else $m_{\rm gal}$ \fi} 
\def \rj {\ifmmode {\cal R}_j \else ${\cal R}_j$ \fi} 
\def \lamgal {\ifmmode \lambda_{\rm gal} \else $\lambda_{\rm gal}$ \fi} 
\def \Vcirc {\ifmmode V_{\rm circ} \else $V_{\rm circ}$ \fi} 
\def \Vrot {\ifmmode V_{\rm rot} \else $V_{\rm rot}$ \fi} 
\def \Vflat {\ifmmode V_{\rm flat} \else $V_{\rm flat}$ \fi} 
\def \Mstar {\ifmmode M_{\rm star} \else $M_{\rm star}$ \fi} 
\def \Mgas {\ifmmode M_{\rm gas} \else $M_{\rm gas}$ \fi} 
\def \Mbar {\ifmmode M_{\rm bar} \else $M_{\rm bar}$ \fi} 

%%MASS-TO-LIGHT RATIOS
\def \DeltaIMF {\ifmmode \Delta_{\rm IMF} \else $\Delta_{\rm IMF}$ \fi}

\def \VV {\ifmmode V_{\rm 2.2}/V_{200} \else $V_{2.2}/V_{200}$ \fi} 
\def \dvr {\ifmmode \partial_{\rm VR} \else $\partial_{\rm VR}$ \fi} 

%%%%%%%%%%%%%%%%%%%%%%%%%%%%%%%%%%%%%%%%%%%%%%%%%%%%%%%%%%%%%%%%%%%%%%

\title[OVI Emssivity] {}

\author[Zhang et al.]{Shuinai Zhang$^{1}$\thanks{snzhang@pmo.ac.cn}, Liang Wang$^{1,3}$,Taotao Fang$^{2}$, Li Ji$^{1}$,
\newauthor{Thales Gutcke$^3$, Andrea V. Macci\`o$^{3,4}$}, Xi Kang$^{1}$\\
$^1$Purple Mountain Observatory, 2 West Beijing Road, Nanjing 210008, China\\
$^2$Xiamen Univeristy \\
$^3$Max-Planck-Institut f\"ur Astronomie, K\"onigstuhl 17, 69117 Heidelberg, Germany\\
$^4$New York University Abu Dhabi, PO Box 129188, Abu Dhabi, UAE}
\begin{document}

\date{to be submitted to }
             
\pagerange{\pageref{firstpage}--\pageref{lastpage}}\pubyear{2016}

\maketitle           

\label{firstpage}
             
%%%%%%%%%%%%%%%%%%%%%%%%%%%%%%%%%%%%%%%%%%%%%%%%%%%%%%%%%%%%%%%%%%%%%%

\begin{abstract}
  We use the NIHAO galaxy formation simulations to study ultra-violet (UV)
  emission from circum-galactic medium (CGM) in galaxies ranging from dwarf
  ($\Mhalo\sim10^{10} \Msun$) to Milky Way ($\Mhalo\sim10^{12} \Msun$)
  masses. We analyze the spatially-extended structures of emission lines
  from OVI and Ly$alpha$.
\end{abstract}

\begin{keywords}
  galaxies: evolution -- galaxies: formation -- galaxies: dwarf -- galaxies: spiral -- 
  methods: numerical -- cosmology: theory
\end{keywords}

\setcounter{footnote}{1}

%%%%%%%%%%%%%%%%%%%%%%%%%%%%%%%%%%%%%%%%%%%%%%%%%%%%%%%%%%%%%%%%%%%%%%
%% SECTION 1: INTRODUCTION
%%%%%%%%%%%%%%%%%%%%%%%%%%%%%%%%%%%%%%%%%%%%%%%%%%%%%%%%%%%%%%%%%%%%%%

\section{Introduction}
\label{sec:intro}

This paper is organized as follows: The cosmological hydrodynamical
simulations and the methodology for computing metal line emission
are briefly described in  \S\ref{sec:method}; In \S\ref{sec:results} 
we present the results including the Ly$alpha$ and OVI emission map,
surface brightness and luminosity evolutions of all galaxies in NIHAO
sample; \S\ref{sec:sum} gives discussion and  summary of our
results.

%%%%%%%%%%%%%%%%%%%%%%%%%%%%%%%%%%%%%%%%%%%%%%%%%%%



%%%%%%%%%%%%%%%%%%%%%%%%%%%%%%%%%%%%%%%%%%%%%%%%%%%
%% SECTION 2 METHODOLOGY
%%%%%%%%%%%%%%%%%%%%%%%%%%%%%%%%%%%%%%%%%%%%%%%%%%%
\section{Methodology} 
\label{sec:method}

\subsection{Simulations}
\label{sec:sims}

The simulations studied in this work are from the NIHAO (Numerical
Investigation of a Hundred Astrophysical Objects) project \citep{Wang15}.  
The halos to be re-simulated with baryons have been extracted from 
3 different pure N-body simulations with a box size of 60, 20 and 
15 $h^{-1}$ Mpc respectively\citet{Dutton14}. 
All halos across the whole mass range with typically a million dark
matter particles inside the virial radius of the target halo at 
redshift $z=0$.  
We adopted the  latest compilation of cosmological
parameters from the Planck  satellite \citep{Planck14}. 

We use the SPH hydrodynamics code {\sc gasoline} \citep{Wadsley04},
with a revised treatment of  hydrodynamics as described in
\citet{Keller14}.  The code includes a subgrid model for turbulent
mixing of metal and energy \citep{Wadsley08}, heating and cooling
include photoelectric heating of dust grains, ultraviolet (UV) heating
and ionization and  cooling due to hydrogen, helium and metals
\citep{Shen10}.  The star formation and feedback modeling follows what
was used in the MaGICC simulations \citep{Stinson13}.  There are two
small changes in NIHAO simulations: The change in  number of neighbors
and the new combination of softening length and  particle mass means
the threshold for star formation increased from  9.3 to 10.3
cm$^{-3}$, the increase of pre-SN feedback efficiency $\epsilon_{\rm
  ESF}$, from 0.1 to 0.13.  The more detail on star formation and
feedback modeling can be found in \citet{Wang15}.

\subsection{Emissivity Calculation}
\label{sec:emis}

We first assign the number densities and temperatures of all gas particles
inside $2 R_{\rm vir}$ to $200\times 200 \times 200$ grids according SPH
spline kernel \citep{Monaghan85}:
\begin{equation}
W\left(r,h\right) = \frac{8}{\pi h^3}
  %\begin{cases}
  %1-6\left(\frac{r}{h}\right)^2+6\left(\frac{r}{h}\right)^3, & \quad 0\leq\frac{r}{h}\leq\frac{1}{2},\\
  %2\left(1-\frac{r}{h}\right)^3, & \quad \frac{1}{2} < \frac{r}{h} \leq 1,\\
  %0, & \quad \frac{r}{h} >1.\\
  %\end{cases}
\end{equation}

The emissivities are computed as a function of gas temperature and 



%%%%%%%%%%%%%%%%%%%%%%%%%%%%%%%%%%%%%%%%%%%%%%%%%%%%%%%%%%%%%%%%%%%%%%%%%%%%
%% SECTION 3 RESULTS
%%%%%%%%%%%%%%%%%%%%%%%%%%%%%%%%%%%%%%%%%%%%%%%%%%%%%%%%%%%%%%%%%%%%%%%%%%%%%%%
\section{Results}
\label{sec:results}





%%%%%%%%%%%%%%%%%%%%%%%%%%%%%%%%%%%%%%%%%%%%%%%%%%%%%%%%%%%%%%%%%%%%%%
%% SECTION 4: SUMMARY
%%%%%%%%%%%%%%%%%%%%%%%%%%%%%%%%%%%%%%%%%%%%%%%%%%%%%%%%%%%%%%%%%%%%%%
\section{Summary}
\label{sec:sum}



\section*{Acknowledgments} 

{\sc Gasoline} was written by Tom Quinn and James Wadsley. Without
their contribution, this paper would have been impossible.
The simulations were performed on the {\sc theo} cluster of the
Max-Planck-Institut f\"ur Astronomie and the {\sc hydra} cluster at
the Rechenzentrum in Garching; and the Milky Way supercomputer, funded
by the Deutsche Forschungsgemeinschaft (DFG) through Collaborative
Research Center (SFB 881) "The Milky Way System" (subproject Z2),
hosted and co-funded by the J\"ulich Supercomputing Center (JSC). We
greatly appreciate the contributions of all these computing
allocations. AVM acknowledge support through the
Sonderforschungsbereich SFB 881 “The Milky Way System” (subproject A1)
of the German Research Foundation (DFG).  The analysis made use of the
pynbody package \citep{Pontzen13}.
%
The authors acknowledge support from the MPG-CAS through the
partnership programme between the MPIA group lead by AVM and the PMO
group lead by XK.
LW acknowledges support of the MPG-CAS student
programme.
XK acknowledge the support from NSFC project
No.11333008 and the ``Strategic Priority Research Program the Emergence
of Cosmological Structures'' of the CAS(No.XD09010000).


%%%%%%%%%%%%%%%%%%%%%%%%%%%%%%%%%%%%%%%%%%%%%%%%%%%%%%%%%%%%%%%%%%%%%%
%%  REFERENCES
%%%%%%%%%%%%%%%%%%%%%%%%%%%%%%%%%%%%%%%%%%%%%%%%%%%%%%%%%%%%%%%%%%%%%% 

\begin{thebibliography}{}

%%AAAAAAAA

% Extended Hot Halos around Isolated Galaxies Observed in the ROSAT All-Sky Survey
\bibitem[Anderson et al.(2013)]{Anderson13} Anderson, M.~E., Bregman, J.~N., Dai, X.\ 2013, \apj, 762, 106

% Fundamental differences between SPH and grid methods
\bibitem[Agertz et al.(2007)]{Agertz07} Agertz, O., Moore, B., Stadel, J.\ 2007, \mnras, 380, 963

%%BBBBBBBB

%The Average Star Formation Histories of Galaxies in Dark Matter Halos
% from z = 0-8
\bibitem[Behroozi et al.(2013)]{Behroozi13} Behroozi, P.~S.,
  Wechsler, R.~H., \& Conroy, C.\ 2013, \apj, 770, 57

%A First Estimate of the Baryonic Mass Function of Galaxies
\bibitem[Bell et al.(2003)]{Bell03} Bell, E.~F., McIntosh, D.~H., Katz, N., Weinberg, M.~D.,\ 2003, \apj, 585, 117

% Multiscale Gaussian Random Fields and Their Application to Cosmological Simulations
%\bibitem[Bertschinger(2001)]{Bertschinger01} Bertschinger, E.\ 2001, 
%\apjs, 137, 1 

% The Search for the Missing Baryons at Low Redshift
\bibitem[Bregman (2007)]{Bregman07}
Bregman, J.~N.\ 2007, ARAA, 45, 221

%% CCCCCCCC

%Where are the baryons
\bibitem[Cen \& Ostriker (1999)]{Cen09} 
Cen, R.~Y., Ostriker, J.~P.\ 1999, \apj, 514, 1



%% DDDDDDDD

%Dark halo response and the stellar initial mass function in early-type and late-type galaxies
\bibitem[Dutton et al.(2011)]{Dutton11} Dutton, A.~A., Conroy, 
  C., van den Bosch, F.~C., et al.\ 2011, \mnras, 416, 322
  
%The baryonic Tully-Fisher relation and galactic outflows
\bibitem[Dutton(2012)]{Dutton12} Dutton, A.~A.\ 2012,
  \mnras, 424, 3123
  
%Cold dark matter haloes in the Planck era: evolution of structural parameters for Einasto and NFW profiles
\bibitem[Dutton \& Macci{\`o}(2014)]{Dutton14} Dutton,
  A.~A., \& Macci{\`o}, A.~V.\ 2014, \mnras, 441, 3359 


%% EEEEEEEEE

    
%% FFFFFFFFF

% The Cosmic Baryon Budget
\bibitem[Fukugita et al.(1998)]{Fukugita98} Fukugita, M., Hogan, C.~J., Peebles, P.~J.~F.\ 1998, \apj, 503, 518

%% GGGGGGGGG




%% HHHHHHHHH



%% JJJJJJJJJJJ


%% KKKKKKKKKKK

%A superbubble feedback model for galaxy simulations
\bibitem[Keller et al.(2014)]{Keller14} Keller, B.~W., Wadsley, 
  J., Benincasa, S.~M., \& Couchman, H.~M.~P.\ 2014, \mnras, 442, 3013

% Stellar mass -- halo mass relation and star formation efficiency in
% high-mass halos
\bibitem[Kravtsov et al.(2014)]{Kravtsov14} Kravtsov, A., 
Vikhlinin, A., \& Meshscheryakov, A.\ 2014, arXiv:1401.7329 

  
%% LLLLLLLLL


%% MMMMMMMMM

% The Baryon Content of Cosmic Structures
\bibitem[McGaugh et al.(2010)]{McGaugh10} McGaugh, S.~S., 
Schombert, J.~M., de Blok, W.~J.~G., Zagursky, M.~J.\ 2010, \mnras,
708, 14

% A refined particle method for astrophysical problems
\bibitem[Monaghan \& Lattanzio(1985)]{Monaghan85} Monaghan, J.~J., Lattanzio, J.~C., 1985, \aap, 149, 135

% Galactic star formation and accretion histories from matching galaxies to dark matter haloes
\bibitem[Moster et al.(2013)]{Moster13} Moster, B.~P., Naab, T., 
\& White, S.~D.~M.\ 2013, \mnras, 428, 3121 





%% NNNNNNNNN




%% OOOOOOOO


%% PPPPPPPPP

%Planck 2013 results. XVI. Cosmological parameters
\bibitem[the Planck Collaboration et  al.(2014)]{Planck14}
  Planck Collaboration, Ade, P.~A.~R., Aghanim, N., et al.\ 2014,
  \aap, 571, AA16 
 
% A Budget and Accounting of Metals at z ~ 0: Results from the COS-Halos Survey
\bibitem[Peeples et al.(2014)]{Peeples14} Peeples, M.~S., Werk, J.~K., Tumlinson, J., et al.\ 2014, \apj, 786, 54

% Dark MaGICC: the effect of dark energy on disc galaxy formation. Cosmology does matter
%\bibitem[Penzo et al.(2014)]{Penzo14} Penzo, C., Macci{\`o}, 
%  A.~V., Casarini, L., Stinson, G.~S., \& Wadsley, J.\ 2014, \mnras, 442, 176

% The baryon content of the universe
\bibitem[Persic \& Salucci(1992)]{Persic92} Persic, M., Salucci, P.\ 1992, \mnras, 258, 14

%pynbody: N-Body/SPH analysis for python
\bibitem[Pontzen et al.(2013)]{Pontzen13} Pontzen, A., Ro{\v s}kar, R., Stinson, G., \& Woods, R.\ 2013, Astrophysics Source Code Library, 1305.002 


%% RRRRRRRRRRR

%Evolution of the Mass-Metallicity Relations in Passive and Star-forming Galaxies from SPH-cosmological Simulations
  
%Multiphase smoothed-particle hydrodynamics
%\bibitem[Ritchie \& Thomas(2001)]{Ritchie01} Ritchie, B.~W.,
%  \& Thomas, P.~A.\ 2001, \mnras, 323, 743

%% SSSSSSSSSSSS

% The enrichment of the intergalactic medium with adiabatic feedback - I. Metal cooling and metal diffusion
\bibitem[Shen et al.(2010)]{Shen10} Shen, S., Wadsley, J., 
\& Stinson, G.\ 2010, \mnras, 407, 1581 

% The Baryon Census in a Multiphase Intergalactic Medium: 30% of the Baryons May Still be Missing
\bibitem[Shull et al.(2012)]{Shull12}
Shull, J.~M., Smith, B.~D., Danforth, C.~W.\ 2012, \apj, 759, 23

%Making Galaxies In a Cosmological Context: the need for early stellar feedback
\bibitem[Stinson et al.(2013)]{Stinson13} Stinson, G.~S., Brook, 
C., Macci{\`o}, A.~V., et al.\ 2013, \mnras, 428, 129 



%% TTTTTTTTTTT

% Not Dead Yet: Cool Circumgalactic Gas in the Halos of Early-type Galaxies
\bibitem[Thom et al.(2012)]{Thom12}
Thom, C., Tumlinson, J., Werk, J.~K.\ 2012, \apjl, 758, L41

% The Large, Oxygen-Rich Halos of Star-Forming Galaxies Are a Major Reservoir
% of Galactic Metals
\bibitem[Tumlinson et al.(2011)]{Tumlinson11} Tumlinson, J., Thom, C., Werk, J., et al.\ 2011, Science, 334, 948

% The COS-Halos Survey: Rationale, Design and a Census of Circumgalactic Neutral Hydrogen
\bibitem[Tumlinson et al.(2013)]{Tumlinson13} Tumlinson, J., Thom, C., Werk, J., et al.\ 2013, \apj, 777, 59

%% VVVVVVVVVVVV



%% WWWWWWWWWWWWWW

%Gasoline: a flexible, parallel implementation of TreeSPH
\bibitem[Wadsley et al.(2004)]{Wadsley04} Wadsley, J.~W., Stadel, 
J., \& Quinn, T.\ 2004, \na, 9, 137 

%On the treatment of entropy mixing in numerical cosmology
\bibitem[Wadsley et al.(2008)]{Wadsley08} Wadsley, J.~W., 
Veeravalli, G., \& Couchman, H.~M.~P.\ 2008, \mnras, 387, 427 

  %NIHAO project I: Reproducing the inefficiency of galaxy formation across cosmic time with a large sample of cosmological hydrodynamical simulations
\bibitem[Wang et al.(2015)]{Wang15} Wang, L., Dutton, A.~A.,  Stinson, G.~S., et al.\ 2015, \mnras, 454, 83
  
% The COS-Halos Survey: Keck LRIS and Magellan MagE Optical Spectroscopy
\bibitem[Werk et al.(2012)]{Werk12} Werk, J.~k., Prochaska, J.~X., Thom, C., et al.\ 2012, \apjs, 198, 3

% The COS-Halos Survey: An Empirical Description of Metal-line Absorption in the Low-redshift Circumgalactic Medium
\bibitem[Werk et al.(2013)]{Werk13} Werk, J.~k., Prochaska, J.~X., Thom, C., et al.\ 2013, \apjs, 204, 17

%The COS-Halos Survey: Physical Conditions and Baryonic Mass in the Low-redshift Circumgalactic Medium
\bibitem[Werk et al.(2014)]{Werk14} Werk, J.~k., Prochaska, J.~X., Thom, C., et al.\ 2014, \apj, 792, 8


%% YYYYYYYYYYYY


%% ZZZZZZZZZZZZZ

\end{thebibliography}

\label{lastpage}
\end{document}
